\usepackage{animate}
\usepackage{media9}
\usepackage{tikz}

\definecolor{color1}{RGB}{56,108,241}
\definecolor{color2}{RGB}{38,164,66}
\definecolor{color3}{RGB}{216,56,56}
\definecolor{color4}{RGB}{149,52,200}
\definecolor{color5}{RGB}{122,43,26}
\definecolor{color6}{RGB}{42,194,189}
\definecolor{color7}{RGB}{126,126,126}
\definecolor{color8}{RGB}{35,35,35}

\newcommand{\back}{\hspace*{-0.13cm}}

\newcommand*{\makeMine}[3]{
    \begin{animateinline}[label=mine#1x#2]{1}
        \tikz{\node[fill=gray,minimum width=1cm,minimum height=1cm] {};}
    \newframe
        \tikz{\node[fill=lightgray,minimum width=1cm,minimum height=1cm] {};
            \draw[thick] (0,0.4)--(0,-0.4) (0.4,0)--(-0.4,0) (0.28,0.28)--(-0.28,-0.28) (0.28,-0.28)--(-0.28,0.28);
            \draw[fill=red] (0,0) circle(0.2);}
    \newframe
        \tikz{\node[fill=gray,minimum width=1cm,minimum height=1cm] {};
            \draw[fill=red] (-0.15,0.3)--(0.2,0.15)--(-0.15,0)--cycle;
            \draw[thick] (-0.15,-0.3)--(-0.15,0.3);}
    \newframe
        \tikz{\node[fill=lightgray,minimum width=1cm,minimum height=1cm] {};}
    \newframe
        \tikz{\node[fill=lightgray,minimum width=1cm,minimum height=1cm,text centered,text=color1] {\LARGE\textbf{1}};}
    \newframe
        \tikz{\node[fill=lightgray,minimum width=1cm,minimum height=1cm,text centered,text=color2] {\LARGE\textbf{2}};}
    \newframe
        \tikz{\node[fill=lightgray,minimum width=1cm,minimum height=1cm,text centered,text=color3] {\LARGE\textbf{3}};}
    \newframe
        \tikz{\node[fill=lightgray,minimum width=1cm,minimum height=1cm,text centered,text=color4] {\LARGE\textbf{4}};}
    \newframe
        \tikz{\node[fill=lightgray,minimum width=1cm,minimum height=1cm,text centered,text=color5] {\LARGE\textbf{5}};}
    \newframe
        \tikz{\node[fill=lightgray,minimum width=1cm,minimum height=1cm,text centered,text=color6] {\LARGE\textbf{6}};}
    \newframe
        \tikz{\node[fill=lightgray,minimum width=1cm,minimum height=1cm,text centered,text=color7] {\LARGE\textbf{7}};}
    \newframe
        \tikz{\node[fill=lightgray,minimum width=1cm,minimum height=1cm,text centered,text=color8] {\LARGE\textbf{8}};}
    \newframe
        \tikz{\node[fill=green,minimum width=1cm,minimum height=1cm] {};
            \draw[fill=red] (-0.15,0.3)--(0.2,0.15)--(-0.15,0)--cycle;
            \draw[thick] (-0.15,-0.3)--(-0.15,0.3);}
    \end{animateinline}%
    \hspace{-1cm}\back%
    \mediabutton[jsaction={
        var dim = #3;
        var num_mines = Math.floor(dim*dim*0.2);
        var mines = anim["mines"];
        if(typeof mines == "undefined" || mines == 0) {
            mines = new Array(dim*dim);
            for(var n = 0; n < dim*dim; n++)
                mines[n] = false;
            for(var n = 0; n < num_mines; n++) {
                var x = Math.floor(dim * Math.random());
                var y = Math.floor(dim * Math.random());
                if(mines[x*dim + y] || (Math.abs(x-#1) <= 1 && Math.abs(y-#2) <= 1))
                    n--;
                else
                    mines[x*dim + y] = true;
            }
            anim["mines"] = mines;
        }
        function uncover(x, y) {
            if(anim["mine" + x + "x" + y].frameNum != 0)
                return;
            if(mines[x*dim + y]) {
                anim["mine" + x + "x" + y].frameNum = 1;
                anim["lost"] = 1;
                return;
            }
            var num = 0;
            for(var i = Math.max(0, x-1); i < Math.min(dim, x+2); i++)
                for(var j = Math.max(0, y-1); j < Math.min(dim, y+2); j++)
                    if(x != i || y != j)
                        if(mines[i*dim+j])
                            num++;
            anim["mine" + x + "x" + y].frameNum=num+3;
            if(num == 0)
                for(var i = Math.max(0, x-1); i < Math.min(dim, x+2); i++)
                    for(var j = Math.max(0, y-1); j < Math.min(dim, y+2); j++)
                        if(x != i || y != j)
                            uncover(i, j);
        }
        if(anim["lost"] != 1) {
            if(anim["flagBox"].frameNum == 1) {
                if(anim["mine" + #1 + "x" + #2].frameNum == 0) anim["mine" + #1 + "x" + #2].frameNum = 2;
                else if(anim["mine" + #1 + "x" + #2].frameNum == 2) anim["mine" + #1 + "x" + #2].frameNum = 0;
                anim["flagBox"].frameNum = 0;
            } else if(anim["mine" + #1 + "x" + #2].frameNum == 0)
                uncover(#1, #2);
            else if(anim["mine" + #1 + "x" + #2].frameNum >= 4) {
                var flag_count = 0;
                for(var x = Math.max(0, #1-1); x < Math.min(dim, #1+2); x++)
                    for(var y = Math.max(0, #2-1); y < Math.min(dim, #2+2); y++)
                        if(anim["mine" + x + "x" + y].frameNum == 2)
                            flag_count++;
                if(flag_count+3 == anim["mine" + #1 + "x" + #2].frameNum)
                for(var x = Math.max(0, #1-1); x < Math.min(dim, #1+2); x++)
                    for(var y = Math.max(0, #2-1); y < Math.min(dim, #2+2); y++)
                        uncover(x, y);
            }
            var found_mines = 0;
            for(var x = 0; x < dim; x++)
                for(var y = 0; y < dim;  y++) {
                    if(anim["mine" + x + "x" + y].frameNum == 2 && mines[x*dim+y])
                        found_mines++;
                    else if(anim["mine" + x + "x" + y].frameNum == 2 || mines[x*dim+y])
                        found_mines--;
                }
            if(found_mines == num_mines) {
                for(var x = 0; x < dim; x++)
                    for(var y = 0; y < dim;  y++)
                        if(anim["mine" + x + "x" + y].frameNum == 2)
                            anim["mine" + x + "x" + y].frameNum = 12;
                anim["lost"] = 1;
            }
        }
    }]{
        \tikz{\node[draw,minimum width=1cm,minimum height=1cm] {};}
    }%
}

% Setup game field
\newcommand{\makeMines}[1]{
    \foreach \i in {0,...,#1-1}{ %
        \foreach \j in {0,...,#1-1}{%
            \makeMine{\i}{\j}{#1} \hspace*{-0.35cm}%
        } \\ \vspace*{-0.1cm}%
    }
    \vspace{0.5cm}
}


% Toggle flag more
\newcommand{\makeFlagButton}{
    \begin{animateinline}[label=flagBox]{1}
        \tikz{\node[fill=gray,minimum width=1cm,minimum height=1cm] {};
            \draw[fill=red] (-0.15,0.3)--(0.2,0.15)--(-0.15,0)--cycle;
            \draw[thick] (-0.15,-0.3)--(-0.15,0.3);}
    \newframe
        \tikz{\node[fill=green,minimum width=1cm,minimum height=1cm] {};
            \draw[fill=red] (-0.15,0.3)--(0.2,0.15)--(-0.15,0)--cycle;
            \draw[thick] (-0.15,-0.3)--(-0.15,0.3);}
    \end{animateinline}%
    \hspace{-1cm}%
    \mediabutton[jsaction={
        if(anim["flagBox"].frameNum == 1)
            anim["flagBox"].frameNum = 0;
        else
            anim["flagBox"].frameNum = 1;
    }]{\tikz{\node[draw,minimum width=1cm,minimum height=1cm] {};}}
}

% Reset game
\newcommand{\makeResetButton}[1]{
    \tikz{\node[fill=gray,minimum width=1cm,minimum height=1cm] {};
        \draw[->,thick,blue](0,-0.3) arc(270:0:0.3);
    }%
    \hspace{-1cm}%
    \mediabutton[jsaction={
        var dim = #1;
        for(var i = 0; i < dim; i++)
            for(var j = 0; j < dim; j++)
                anim["mine" + i + "x" + j].frameNum = 0;
        anim["mines"] = 0;
        anim["lost"] = 0;
    }]{\tikz{\node[draw,minimum width=1cm,minimum height=1cm] {};}}
}

% options Javascript debug window in Adobe Reader
\newcommand{\makeDebugButton}{
    \mediabutton[jsaction={
        non_existing_function();
    }]{\fbox{Debug}}
}
